\documentclass[12pt]{article}
% fleqn: Causes displayed formulas to be left-aligned.
% leqno: For numbered displayed formulas, the number would be put to the left side. The right side is the default.


\usepackage[english, ukrainian]{babel}
% This is useful especially for typesetting in other languages than English regarding hyphenation, language-specific characters, and more.


\usepackage[utf8]{inputenc}
% The option utf8 tells the package to use Unicode input encoding, which provides many more symbols than just the ASCII code. Now we just need to find the symbol on the keyboard and to type it.
% takes care of the input and translates special characters into TeX macros. It understands an option to mark the encoding. This depends on the editor and the operating system. As a rule of thumb: on Linux and Mac OS X, use utf8; on Windows, use latin1, except with TeXworks. Also, on Windows, TeXworks understands utf8, which is an implementation of Unicode.


\usepackage[T1]{fontenc}
% is responsible for the output encoding: TeX macros are translated into special characters. For instance, ö is no longer constructed of o and dots, but it's the glyph ö of the current font. Thus, hyphenation rules can be applied, the search feature of a PDF reader works with those characters, even copy and paste works fine. With standard encoding, copying and pasting ö would result in dots and an o.


%\usepackage{parskip}
% Its only purpose is to remove the paragraph indentation completely. At the same time, this package introduces a skip between paragraphs.


\usepackage{geometry}
% This package takes care of our layout regarding the paper size, margins, and more dimensions.


\usepackage{blindtext}
\usepackage{layouts}


\usepackage{xifthen}
% for \iftenelse{condition}{then}{else}


\usepackage{amsmath}
% Because we loaded the amsmath package, we have access to several multi-line math environments. Each line in such an environment is ended by \\, except the last one. The alignment depends on the environment, as we've seen: 
% multline: The first line is left-aligned, the last line right-aligned, and all other lines in between are centered 
% gather: All lines are centeredalign: The lines are aligned at marked relation signs
% align: The lines are aligned at marked relation signs


\usepackage{amsthm}


\usepackage{amsfonts} % to comment


\usepackage{amssymb}


\usepackage{mathtools}
% The mathtools package extends amsmath. If you need a certain feature and cannot find it, neither in standard LaTeX nor in amsmath, always look first at mathtools


\usepackage{array}
\usepackage{booktabs}
% Use \toprule, \midrule, and \bottomrule instead of \hline. Specify a thickness as an optional argument
\usepackage{multirow}
% enchanhing tabular envroment
\usepackage{tabularx}
% environment tabularx requires an additional argument: the width of the table. It introduces a new column type X. X columns behave like p columns, but they use all available space. One X column would take all of the available space. If you use several X columns, they would share the space equally. So you could write, for instance: \begin{tabularx}{0.6\textwidth}{lcX}. This way you would get a table occupying 60 percent of the text width, a left aligned and a centered column as wide as their content, and a paragraph column as wide as possible until 60 percent is reached. LaTeX provides a starred version of the tabular environment: \begin{tabular*}{width}[position]{column specifiers} The table is set to width, but by modifying the inter-column space. tabularx has been developed satisfying the need for a more useful way.
% \usepackage{ltxtable}


\usepackage[]{graphicx}
% You might wish to include pictures, diagrams, or drawings made with other programs


\usepackage[dvipsnames]{xcolor}
% for colors setup


\usepackage{float}
% The float package provides a convenient and consistent looking approach. It introduces the placement option H causing the float to appear right there


\usepackage{wrapfig}
% Though it's a bit playful, you might wish to let text flow around a table or a figure. This can be achieved using the wrapfig package and its environments wrapfigure and wraptable.


\usepackage{caption}
% you could enhance the visual appearance of all of your captions


\usepackage{enumitem}
% that is better alternative to paralist package (see \usepackage{paralist} for context)
% possible labels: \arabic*, \alph*, \Alph*, \roman*, \Roman*


\usepackage{xspace}
% provides the command \xspace that inserts a space depending on the following character: If a dot, a comma, an exclamation, or a quotation mark follows, it won't insert a space, but if a normal letter follows, then it will


\usepackage{microtype}
% It introduces font expansion to tweak the justification and uses hanging punctuation to improve the optical appearance of the margins. This may reduce the need of hyphenation and improves the "grayness" of the output


\usepackage[singlespacing]{setspace}
% Its only purpose is to adjust the line spacing.


\usepackage{fancyhdr}
% enchange the look of header and footer


\usepackage{tcolorbox}
% powerful package to change text background


\usepackage{tocbibind}
% automatically add table of images/tables entries to toc if there any


\usepackage[cachedir=minted_package_cache]{minted}
% powerful package for code highlighting


\usepackage{biblatex}


\usepackage{titlesec}
% provides a comprehensive interface for customizing headings, of parts, chapters, sections, and even smaller sectioning parts down to subparagraphs


\usepackage{imakeidx}
% mark places in the text where these concepts occur. Finally, we will order LaTeX to typeset the index


\usepackage{hyperref}
% your document will be hyperlinked as much as possible
\usepackage{cleveref}
% cleveref automatically determines the type of cross-reference and the context in which it's used.


%\usepackage{paralist}
% The used package paralist provides several new list environments designed to be typeset within paragraphs or in a very compact look. 
%%% Numbered lists:
% compactenum: Compact version of the enumerate environment without any vertical space before or after the list or its items
% inparaenum: An enumerated list typeset within a paragraph
% asparaenum: Every list item is formatted like a separate common LaTeX paragraph, but numbered
%%% Bulleted lists:
% compactitem: Compact version of the itemize environment like compactenum
% inparaitem: An itemized list typeset within a paragraph, rarely seen in print
% asparaitem: Like asparaenum, but with symbols instead of numbers

%%% Preamble made dy Vladyslav Rehan 
% https://tex.stackexchange.com/users/283011/vladyslav-rehan
% https://github.com/Dolfost