% !TeX root = ../../../homework.tex
\section{Обчислити визначник}
Обчислити визначник наступними методами:
\begin{itemize}
	\item Зведення до трикутного вигляду
	\item Розклад за елементами деякого рядка або стовпця
\end{itemize}
\begin{gather}
	A=\begin{bmatrix*}[r]
		1&2&-1&1\\
		0&-1&1&2\\
		-1&1&1&-1\\
		2&-1&0&0
	\end{bmatrix*}.
\end{gather}
\begin{dfn}[Визначник трикутної матриці]
	Визначник трикутної матриці рівний добутку елементів головної діагоналі.\label{dfn:diagonal_determiner}
\end{dfn}
\begin{thm}[Лапласа]
		\label{thm:laplas}
	Визначник квадратної матриці розмірності \(n\)
	\begin{gather}
		\begin{bmatrix*}[c]
			a_{1,1}&a_{1,2}&a_{1,3}&\dots&a_{1,n-1}& a_{1,n}\\
			a_{2,1}&a_{2,2}&a_{2,3}&\dots&a_{2,n-1}& a_{2,n}\\
			a_{3,1}&a_{3,2}&a_{3,3}&\dots&a_{3,n-1}& a_{3,n}\\
			\dots&\dots&\dots&\dots&\dots&\dots\\
			a_{n-1,1}&a_{n-1,2}&a_{n-1,3}&\dots&a_{n-1,n-1}&a_{n-1,n}\\
			a_{n,1}&a_{n,2}&a_{n,3}&\dots&a_{n,n-1}& a_{n,n}\\
		\end{bmatrix*}
	\end{gather}
	рівний
	\begin{gather}
		a_{i,1}(-1)^{(i+1)}A_{i,1} + a_{i,2}(-1)^{(i+2)}A_{i,2}  + \dots + a_{i,n-1}(-1)^{(i+n-1)}A_{i,n-1} + a_{i,n}(-1)^{(i+n)}A_{i,n},
	\end{gather}
	або
	\begin{gather}
		a_{1,j}(-1)^{(i+1)}A_{1,j} + a_{2,i}(-1)^{(i+2)}A_{2,i}  + \dots + a_{n-1,i}(-1)^{(i+n-1)}A_{n-1,i} + a_{n,i}(-1)^{(i+n)}A_{n,i}.
	\end{gather}
	Де \(A_{i,j}\) це відповідне алгебраїчне доповнення, та \(i,j\in \mathbb{N},\ i,j \leq n\).
\end{thm}
\subsection{Трикутний вигляд}
\solving
Зведемо матрицю \(A\) до трикутного вигляду методом елементарних перетворень.
\begin{gather}
		A=\begin{bmatrix*}[r]
		1&2&-1&1\\
		0&-1&1&2\\
		-1&1&1&-1\\
		2&-1&0&0
	\end{bmatrix*}_{\substack{\mathrm{III-II}\hfill\\\mathrm{IV-2I}\hfill}}\begin{bmatrix*}[r]
	1&2&-1&1\\
	0&-1&1&2\\
	0&3&0&-2\\
	0&-5&2&2
\end{bmatrix*}_{\substack{\mathrm{III+3II}\hfill\\\mathrm{IV-5II}}}\sim
\end{gather}
\begin{gather}
\sim\begin{bmatrix*}[r]
1&2&-1&1\\
0&-1&1&2\\
0&0&3&4\\
0&0&-3&-8
\end{bmatrix*}_{\mathrm{IV+3III}}\sim\begin{bmatrix*}[r]
1&2&-1&1\\
0&-1&1&2\\
0&0&3&4\\
0&0&0&-4
\end{bmatrix*}.
\end{gather}
За \cref{dfn:diagonal_determiner} маємо:
\begin{gather}
	\det A =1\cdot (-1)\cdot 3\cdot (-4)=12.
\end{gather}
\subsection{Розклад за елементами рядка}
\solving
Знайдемо визначник матриці \(A\) за теоремою \ref{thm:laplas} за рядком. Нехай \(i=4\). Тоді:
\begin{gather}
	\det A = 2\cdot (-1)^5\begin{vmatrix*}[r]
		2&-1&1\\
		-1&1&2\\
		1&1&-1
	\end{vmatrix*}-1\cdot(-1)^6\begin{vmatrix*}[r]
	1&-1&1\\
	0&1&2\\
	-1&1&-1
\end{vmatrix*}=-2(-5)-(-2)=12.
\end{gather}

\section{Довести, що матриця задовольняє рівняння}
Довести, що матриця \(A\) задовольняє рівняння \(f(x)=0\):
\begin{gather}
	A=\begin{bmatrix*}[r]
		7&1\\4&2
	\end{bmatrix*},\\ f(x)=x^2-9x+10.\label{q2:1}
\end{gather}
\solving
Підставимо матрицю \(A\) рівняння \eqref{q2:1}:
\begin{gather}
	f(A)=\begin{bmatrix*}[r]
		7&1\\4&2
	\end{bmatrix*}\cdot\begin{bmatrix*}[r]
	7&1\\4&2
\end{bmatrix*}-9\cdot\begin{bmatrix*}[r]
7&1\\4&2
\end{bmatrix*}+10\begin{vmatrix*}[r]
1&0\\0&1
\end{vmatrix*}=\\
=\begin{bmatrix*}[c]
	49+4&7+2\\
	28+8&4+4
\end{bmatrix*}-\begin{bmatrix*}[c]
63&9\\
36&18
\end{bmatrix*}+\begin{bmatrix*}[r]
10&0\\0&10
\end{bmatrix*}=\\
=\begin{bmatrix*}[r]
	-10&0\\0&-10
\end{bmatrix*}+\begin{bmatrix*}[r]
10&0\\0&10
\end{bmatrix*}=\begin{bmatrix*}[c]
0&0\\0&0
\end{bmatrix*}=0.
\end{gather}
Так як \(f(A)=0\), то матриця \(A\) задовольняє рівняння \(f(x)=0\).

\section{Обернена матриця}
Для матриць \(A\) та \(B\) знайти обернену матрицю та перевірити.
\begin{gather}
	A=\begin{bmatrix*}[r]
		1&4\\2&-7
	\end{bmatrix*},\\
	B=\begin{bmatrix*}[r]
		1&1&-1\\
		2&1&0\\
		1&-1&1
	\end{bmatrix*}.
\end{gather}
\subsection[Матриця A]{Матриця \(A\)}
\solving
Знайдемо обернену матрицю \(A^{-1}\) методом елементарних перетворень.
\begin{gather}
	A=\left[\begin{array}{@{}rr|rr@{}}
		-1&4&1&0\\
		2&-7&0&1
	\end{array}\right]_{\substack{\mathrm{II+2I}}}\sim\left[\begin{array}{@{}rr|rr@{}}
	-1&4&1&0\\
	0&1&2&1
\end{array}\right]_{\mathrm{I-4II}}\sim\\
\sim\left[\begin{array}{@{}rr|rr@{}}
-1&0&-7&-4\\
0&1&2&1
\end{array}\right]_{\mathrm{I\cdot(-1)}}\sim\left[\begin{array}{@{}rr|rr@{}}
1&0&7&4\\
0&1&2&1
\end{array}\right],\ A^{-1}=\begin{bmatrix*}[r]
7&4\\2&1
\end{bmatrix*}.
\end{gather}
Для перевірки, помножимо матрицю \(A\) на \(A^{-1}\):
\begin{gather}
	A\cdot A^{-1}=\begin{bmatrix*}[r]
		1&4\\2&-7
	\end{bmatrix*}\begin{bmatrix*}[r]
	7&4\\2&1
\end{bmatrix*}=\begin{bmatrix*}[r]
1&0\\0&1
\end{bmatrix*}.
\end{gather}
Отримали одиничну матрицю, або матрицю натурального базису. Тобто, якби матриця \(A\) була лінійним оператором та змінювала б простір, то матриця \(A^{-1}\) повертал б всі точки та вектори які були перетворені матрицею \(A\) в їхнє початкове положення.

\subsection[Матриця B]{Матриця \(B\)}
\solving
Знайдемо обернену матрицю \(B^{-1}\) методом елементарних перетворень.
\begin{gather}
	B=\left[\begin{array}{@{}rrr|rrr@{}}
		1&1&-1&1&0&0\\
		2&1&0&0&1&0\\
		1&-1&1&0&0&1
	\end{array}\right]_{\substack{\mathrm{II-2I}\\\hfill\mathrm{III-I}}}\sim\left[\begin{array}{@{}rrr|rrr@{}}
	1&1&-1&1&0&0\\
	0&-1&2&-2&1&0\\
	0&-2&2&-1&0&1
\end{array}\right]_{\mathrm{III-2II}}\sim\\
\sim\left[\begin{array}{@{}rrr|rrr@{}}
	1&1&-1&1&0&0\\
	0&-1&2&-2&1&0\\
	0&0&-2&3&-2&1
\end{array}\right]_{\mathrm{III\cdot\left(-\frac{1}{2}\right)}}\sim\left[\begin{array}{@{}rrr|rrr@{}}
1&1&-1&1&0&0\\
0&-1&2&-2&1&0\\
0&0&1&-\frac{3}{2}&1&\frac{1}{2}
\end{array}\right]_{\substack{\mathrm{I+III}\hfill\\\mathrm{II-2III}}}\sim
\end{gather}
\begin{gather}
\sim\left[\begin{array}{@{}rrr|rrr@{}}
	1&1&0&-\frac{1}{2}&1&-\frac{1}{2}\\
	0&-1&0&1&-1&1\\
	0&0&1&-\frac{3}{2}&1&\frac{1}{2}
\end{array}\right]_{\mathrm{I+III}}\sim\left[\begin{array}{@{}rrr|rrr@{}}
1&0&0&\frac{1}{2}&0&\frac{1}{2}\\
0&-1&0&1&-1&1\\
0&0&1&-\frac{3}{2}&1&\frac{1}{2}
\end{array}\right]_{\mathrm{II\cdot(-1)}}\sim\\
\sim\left[\begin{array}{@{}rrr|rrr@{}}
	1&0&0&\frac{1}{2}&0&\frac{1}{2}\\
	0&1&0&-1&1&-1\\
	0&0&1&-\frac{3}{2}&1&\frac{1}{2}
\end{array}\right],\ B^{-1}=\begin{bmatrix*}[r]
\frac{1}{2}&0&\frac{1}{2}\\
-1&1&-1\\
-\frac{3}{2}&1&\frac{1}{2}
\end{bmatrix*}.
\end{gather}
Для перевірки, помножимо матрицю \(B\) на \(B^{-1}\):
\begin{gather*}
	B\cdot B^{-1}=\begin{bmatrix*}[r]
		1&1&-1\\
		2&1&0\\
		1&-1&1
	\end{bmatrix*}\begin{bmatrix*}[r]
	\frac{1}{2}&0&\frac{1}{2}\\
	-1&1&-1\\
	-\frac{3}{2}&1&\frac{1}{2}
\end{bmatrix*}=\begin{bmatrix*}[c]
\frac{1}{2}-1+\frac{3}{2}&1-1&\frac{1}{2}-1+\frac{1}{2}\\
1-1&1&1-1\\
\frac{1}{2}+1-\frac{3}{2}&-1+1&\frac{1}{2}+1-\frac{1}{2}
\end{bmatrix*}=\begin{bmatrix*}[r]
1&0&0\\
0&1&0\\
0&0&1
\end{bmatrix*}.
\end{gather*}

\section{Система рівнянь}
Розв'язати систему рівнянь \(S\)
\begin{itemize}
	\item Формулами Крамера.
	\item Методом Гауса.
\end{itemize}
\begin{gather}
	S=\begin{cases}
		x-3y+4z=10,\\
		3x-6y+z=0,\\
		x+3y+3z=3.
	\end{cases}.
\end{gather}
\begin{thm}[Метод Крамера]
	\label{thm:kramer}
	Корені квадратної системи
	\begin{gather}
		\begin{cases}
			a_{1,1}x_1+ a_{1,2}x_2+ a_{1,3}x_3+ \dots + a_{1,n-1}x_{n-1}+ a_{1,n}x_n=b_1,\\ 
			a_{2,1}x_1+ a_{2,2}x_2+ a_{2,3}x_3+ \dots + a_{2,n-1}x_{n-1}+ a_{2,n}x_n=b_2,\\
			\ \ \dots\ \ + \  \ \dots  \ +\ \ \dots   \ \ + \dots +\ \  \ \ \dots \ \ \ \ \ + \ \ \dots \ \  =\dots,\\
%			a_{n-1,1}x_1+ a_{n-1,2}x_2+ a_{n-1,3}x_3+ \dots + a_{n-1,n-1}x_{n-1}+ a_{n-1,n}x_n,\\
			a_{n,1}x_1+ a_{n,2}x_2+ a_{n,3}x_3+ \dots + a_{n,n-1}x_{n-1}+ a_{n,n}x_n=b_n,\\
		\end{cases}
	\end{gather}
	можна знайти за формулою
	\begin{gather}
		x_i=\frac{\begin{vmatrix*}[r]
			a_{1,1} & \dots & b_1 & \dots & a_{1,n}\\
			a_{2,1} & \dots & b_2 & \dots & a_{2,n}\\
			\dots   & \dots & \dots & \dots & \dots\\
			a_{n,1} & \dots & b_n & \dots & a_{n,n}\\
		\end{vmatrix*}}{\begin{vmatrix*}[r]
			a_{1,1} & a_{1,2} & a_{1,3} & \dots & a_{1,n}\\
		a_{2,1} & a_{2,2} & a_{2,3} & \dots & a_{2,n}\\
		\dots   & \dots & \dots & \dots & \dots\\
		a_{n,1} & a_{n,2} & a_{n,3} & \dots & a_{n,n}\\
	\end{vmatrix*}}.
	\end{gather}
Де \(1\leq i \leq n\) це номер стовпця в якому знаходяться \(b_{1,2,3\dots}\).
\end{thm}
\subsection{Формули Крамера}
\solving
Знайдемо всі необхідні визначники:
\begin{align}
	&\Delta=\begin{vmatrix*}[r]
		1&-3&4\\
		3&-6&1\\
		1&3&3
	\end{vmatrix*}=63,\\
	&\Delta_1=\begin{vmatrix*}[r]
		10&-3&4\\
		0&-6&1\\
		13&3&3
	\end{vmatrix*}=63,\\
&\Delta_2=\begin{vmatrix*}[r]
	1&10&4\\
	3&0&1\\
	1&13&3
\end{vmatrix*}=63,\\
&\Delta_3=\begin{vmatrix*}[r]
	1&-3&10\\
	3&-6&0\\
	1&3&13
\end{vmatrix*}=189.
\end{align}
За теоремою \ref{thm:kramer} маємо:
\begin{align}
	&x_1=\frac{\Delta_1}{\Delta}=\frac{63}{63}=1,\\
	&x_2=\frac{\Delta_2}{\Delta}=\frac{63}{63}=1,\\
	&x_3=\frac{\Delta_3}{\Delta}=\frac{189}{63}=3.
\end{align}
\subsection{Метод Гауса}
\solving
Розв'яжемо систему методом Гауса:
\subparagraph{Прямий хід Гауса}
\begin{gather}
	\begin{cases}
		x-3y+4z=10,\\
		3x-6y+z=0,\\
		x+3y+3z=3.
	\end{cases}\sim\left[\begin{array}{@{}rrr|r@{}}
	1&-3&4&10\\
	3&-6&1&0\\
	1&3&3&13
\end{array}\right]_{\substack{\mathrm{II-3I}\hfill\\\mathrm{III-I}}}\sim\left[\begin{array}{@{}rrr|r@{}}
1&-3&4&10\\
0&3&-11&-30\\
0&6&-7&3
\end{array}\right]_{\substack{\mathrm{III-2II}}}\sim\\
\sim\left[\begin{array}{@{}rrr|r@{}}
	1&-3&4&10\\
	0&3&-11&-30\\
	0&0&21&63
\end{array}\right]\sim\begin{cases}
x-3y+4z=10,\\
0x+3y-11z=-30,\\
0x+0y+21z=63.
\end{cases}
\end{gather}
\subparagraph{Зворотній хід Гауса}
\begin{gather}
	\begin{cases}
		x-3y+4z=10,\\
		0x+3y-11z=-30,\\
		0x+0y+21z=63.
	\end{cases}\Leftrightarrow\begin{cases}
	x-3y+4z=10,\\
	y=1,\\
	z=3.
\end{cases}\Leftrightarrow\begin{cases}
x=1,\\
y=1,\\
z=3.
\end{cases}
\end{gather}

\section{Дослідити на сумісність}
Дослідити на сумісність систему лінійних алгебричних рівнянь \(S\). Знайти фундаментальну систему розв'язків відповідної однорідної системи. Знайти загальний розв'язок неоднорідної системи за формулою:
\begin{gather}
	X_{\grave{i}}=X_{\hat{i}}+X_{\check{i}},
\end{gather}
де \(X_{\grave{i}}\) - загальний розв'язок неоднорідної системи, \(X_{\hat{i}}\) - загальний розв'язок однорідної системи, \(X_{\check{i}}\) - частинний розв'язок неоднорідної системи.
\begin{gather}
	S=\begin{cases}
		x_1+2x_2-3x_3+x_4=1,\\
		x_1-x_2+x_3-3x_4=2,\\
		3x_1+x_2-2x_3-2x_4=3,\\
		x_1-3x_2+4x_3-4x_4=1.
	\end{cases}
\end{gather}
\begin{thm}[Кронекера-Капеллі]
	\label{thm:kronokerkapeli}
	Система лiнiйних алгебраїчних рiвнянь сумiсна тодi i тiльки тодi, коли ранг розширеної матрицi системи дорiвнює рангу основної матрицi системи, тобто \(\rank(A)=\rank(\bar{A})\). При цьому, якщо \(\rank(A)=\rank(\bar{A})=n\), то система має єдиний розв'язок. Якщо \(\rank(A)=\rank(\bar{A})<n\), то система має безлiч розв'язкiв.
\end{thm}
\begin{dfn}[Класифікація коренів СЛАР]
	\label{dfn:roots}
	Сумiсна система називається {\bfseries визначеною}, якщо вона має єдиний розв'язок, i {\bfseries невизначеною}, якщо вона має бiльше одного розв'язку. В останньому випадку кожен її розв'язок називається {\bfseries частинним розв'язком}. Сукупнiсть частинних розв'язкiв системи називається {\bfseries загальним розв'язком} цiєї системи.
\end{dfn}
\solving
Запишемо матрицю \(A\) та розширену матрицю \(A\) \(\bar{A}\) системи \(S\):
\begin{gather}
	A=\begin{bmatrix*}[r]
		1&2&-3&1\\
		2&-1&1&-3\\
		3&1&-2&-2\\
		1&-3&4&-4
	\end{bmatrix*},\
	S\sim\bar{A}=\left[\begin{array}{@{}rrrr|r@{}}
		1&2&-3&1&1\\
		2&-1&1&-3&2\\
		3&1&-2&-2&3\\
		1&-3&4&-4&1
	\end{array}\right].
\end{gather}
Знайдемо ранг матриці \(\bar{A}\):
\begin{gather}
	\bar{A}=\left[\begin{array}{@{}rrrr|r@{}}
		1&2&-3&1&1\\
		2&-1&1&-3&2\\
		3&1&-2&-2&3\\
		1&-3&4&-4&1
	\end{array}\right]_{\substack{\mathrm{II-2I}\hfill\\\mathrm{III-3I}\hfill\\\mathrm{IV-I}\hfill}}\sim\left[\begin{array}{@{}rrrr|r@{}}
	1&2&-3&1&1\\
	0&5&7&-5&0\\
	0&5&7&-5&0\\
	0&5&7&-5&0
\end{array}\right]\sim\left[\begin{array}{@{}rrrr|r@{}}
1&2&-3&1&1\\
0&5&7&-5&0
\end{array}\right],\\
\rank(\bar{A})=\rank(A)=2<n=4.\label{eq:4:1}
\end{gather}
Згідно з \eqref{eq:4:1} та теоремою \ref{thm:kronokerkapeli} система \(S\) сумісна та невизначена.

Нехай
\begin{gather}
	\mathbb{B}=\{x_1,x_2\},\\
	\mathbb{F}=\{x_3,x_4\},\\
	x_3=c_1,\ x_4=c_2,\ c_i\in \mathbb{R}.
\end{gather}
Тоді:
\begin{gather}
	S\sim\begin{cases}
		x_1+2x_2-3c_1+c_2=1,\\
		-5x^2+7c_1-5c_2=0,\\
		x_3=c_1,\\
		x_4=c_2.
	\end{cases}\sim\begin{cases}
	x_1=1+3c_1+c_2-2x_2,\\
	x_2=\frac{7}{5}c_1-c_2,\\
	x_3=c_1,\\
	x_4=c_2.
\end{cases}\\
x_1=1+3c_1+c_2-\frac{14}{5}c_1-2c_2=1+\frac{c_1}{5}-c_2,\\
S\sim\begin{cases}
	x_1=1+\frac{c_1}{5}-c_2,\\
	x_2=\frac{7}{5}c_1-c_2,\\
	x_3=c_1,\\
	x_4=c_2.
\end{cases}
\end{gather}
Тепер можемо записати наступне:
\begin{gather}
	X=\begin{bmatrix*}[r]
		x_1\\
		x_2\\x_3\\x_4
	\end{bmatrix*}=\begin{bmatrix*}[c]
	1+\frac{c_1}{5}-c_2\\
	\frac{7}{5}c_1-c_2\\
	c_1\\
	c_2
\end{bmatrix*}=\begin{bmatrix*}[c]
\frac{c_1}{5}\\\frac{7c_1}{5}\\c_1\\0
\end{bmatrix*}+\begin{bmatrix*}[r]
-c_2\\-c_2\\0\\c_2
\end{bmatrix*}+\begin{bmatrix*}[r]
1\\0\\0\\0
\end{bmatrix*}=\frac{c_1}{5}\begin{bmatrix*}[r]
1\\7\\1\\0
\end{bmatrix*}+c_2\begin{bmatrix*}[r]
-1\\-1\\0\\1
\end{bmatrix*}+\begin{bmatrix*}[r]
1\\0\\0\\0
\end{bmatrix*}=\\
=c_1^\circ\begin{bmatrix*}[r]
	1\\7\\1\\0
\end{bmatrix*}+c_2\begin{bmatrix*}[r]
	-1\\-1\\0\\1
\end{bmatrix*}+\begin{bmatrix*}[r]
	1\\0\\0\\0
\end{bmatrix*}=c_1^\circ e_1+c_2e_2+\tilde{e}.
\end{gather}
Де:
\begin{align}
	&c_1^\circ e_1+c_2e_2+\tilde{e}\ - \text{ загальний розв'язок неоднорідної системи},\\
	&e_1,\ e_2\ - \text{ фундаментальна система розв'язків},\\
	&\tilde{e}\ - \text{ частинний розв'язок неоднорідної системи}.
\end{align}

\section{Дано координати точок. Знайти\dots}
Дано координати точок \(M_1,\ M_2,\ M_3\) та \(M_4\)
\begin{align}
	&M_1(2;1;4),\\
	&M_2(-1;5;-1),\\
	&M_3(-7;-3;2),\\
	&M_4(-6;-3;6).
\end{align}
Знайти:
\begin{enumerate}
	\item Довжину та напрямні косинуси вектора \(\vv{M_1M_2}\).
	\item Кут \(\angle M_2M_1M_3\).
	\item Площу \(\triangle M_1M_2M_3\).
	\item Об'єм піраміди \(M_1M_2M_3M_4\).
	\item Висоту піраміди \(M_4H\), скориставшись проекцією вектора на вісь.
\end{enumerate}
\subsection{Довжина та напрямні косинуси}
\solving
Найдемо координати вектора \(\vv{M_1M}_2\):
\begin{gather}
	\vv{m}=(-3,4,-5),\\
	\lvert\vv{m}_1\rvert=\sqrt{(-3)^2+4^2+(-5)^2}=\sqrt{50}=5\sqrt{2}.\\
\end{gather}
Зайдемо напрямні косинус вектора \(\vv{m}_1\)
\begin{gather}
	\cos\alpha=\frac{m_{1x}}{\lvert\vv{m}_1\rvert}=\frac{-3}{5\sqrt{2}}=-\frac{3\sqrt{2}}{10},\\
	\cos\beta=\frac{m_{1y}}{\lvert\vv{m}_1\rvert}=\frac{4}{5\sqrt{2}}=\frac{2\sqrt{2}}{5},\\
	\cos\gamma=\frac{m_{1z}}{\lvert\vv{m}_1\rvert}=\frac{-5}{5\sqrt{2}}=-\frac{\sqrt{2}}{2}.
\end{gather}
\subsection{Кут}
\solving
Знайдемо вектори \(\vv{M_2M}_1=\vv{m}_1\) та \(\vv{M_1M}_3=\vv{m}_2\):
\begin{gather}
	\vv{m}_1=(3;-4;5),\ \vv{m}_2=(-9;-4;-2).
\end{gather}
Знайдемо довжини векторів \(\vv{m_1}\) та \(\vv{m}_2\):
\begin{gather}
	\lvert\vv{m}_1\rvert=\sqrt{9+16+25}=5\sqrt{2},\\
	\lvert\vv{m}_2\rvert=\sqrt{81+16+4}=\sqrt{101}.
\end{gather}
Знайдемо кут \(\alpha=\angle\widehat{\vv{m}_1\vv{m}_2}\):
\begin{gather}
	\cos\alpha=\frac{\vv{m}_1\cdot\vv{m}_2}{\lvert\vv{m}_1\rvert\lvert\vv{m}_1\rvert}=\frac{-27+16-10}{5\sqrt{202}}=\frac{-21}{5\sqrt{202}},\\
	\alpha=\arccos\frac{-21}{5\sqrt{202}}.
\end{gather}
\subsection{Площа}
\solving
Знайдемо вектори \(\vv{M_1M}_2=\vv{m}_1\) та \(\vv{M_1M}_3=\vv{m}_2\):
\begin{gather}
	\vv{m}_1=(-3;4;-5),\ \vv{m}_2=(-9;-4;-2).
\end{gather}
Знайдемо довжину \(\vv{m}_{1,2}\):
\begin{gather}
		\lvert\vv{m}_1\rvert=\sqrt{9+16+25}=5\sqrt{2},\\
	\lvert\vv{m}_2\rvert=\sqrt{81+16+4}=\sqrt{101}.
\end{gather}
Знайдемо косинус \(\angle\widehat{\vv{m}_1\vv{m}_2}\):
\begin{gather}
	\cos\widehat{\vv{m}_1\vv{m}_2}=\frac{\vv{m}_1\cdot\vv{m}_2}{\lvert\vv{m}_1\rvert\lvert\vv{m}_1\rvert}=\frac{27-16-10}{5\sqrt{202}}=\frac{1}{5\sqrt{202}}.
\end{gather}
Знайдемо синус \(\angle\widehat{\vv{m}_1\vv{m}_2}\):
\begin{gather}
	\sin\widehat{\vv{m}_1\vv{m}_2}=\sqrt{1-\cos^2\widehat{\vv{m}_1\vv{m}_2}}=\sqrt{1-\frac{1}{25\cdot202}}=\sqrt{\frac{5049}{5050}}=\frac{3\sqrt{561}}{5\sqrt{202}}.
\end{gather}
Знайдемо площу \(\triangle\):
\begin{gather}
	S_\triangle=\frac{1}{2}\lvert\vv{m}_1\rvert\lvert\vv{m}_2\rvert\sin\widehat{\vv{m}_1\vv{m}_2}
	=\frac{15\sqrt{202\cdot 561}}{10\sqrt{202}}=\frac{3\sqrt{561}}{2}.
\end{gather}
\subsection{Об'єм}
\solving
Знайдемо координати векторів \(\vv{M_1M}_2=\vv{m}_1,\ \vv{M_2M}_3=\vv{m_2}\) та \(\vv{M_3M}_4=\vv{m}_3\):
\begin{gather}
	\vv{m}_1=(-3;4;-5),\ \vv{m}_2=(-6;-8;3),\ \vv{m}_3=(1;0;4).
\end{gather}
Знайдемо об'єм піраміди \(M_1M_2M_3M_4\):
\begin{gather}
	V\triangle=\frac{1}{6}\lvert(\vv{m}_1\times \vv{m}_2)\cdot\vv{m}\rvert=\frac{1}{6}\begin{vmatrix*}[r]
		-3&4&-5\\
		-6&-8&3\\
		1&0&4
	\end{vmatrix*}=\frac{1}{6}164=27\frac{1}{3}.
\end{gather}

\subsection{Висота}


\section{Дано вектори\dots}
Дано вектори 
\begin{align}
	&\vv{p} = (0;5;1),\\ 
	&\vv{q} = (3;2;-1),\\ 
	&\vv{r} = (-1;1;1).
\end{align}
Довести, що вони утворюють базис. Знайти:
\begin{enumerate}
	\item Координати вектора \(\vv{a}=(-15;5;6)\) в цьому базисі.
	\item Вектор \(\vv{c}\), який ортогональних до векторів \(\vv{p}\) і \(\vv{q}\), якщо вектор \(\vv{c}\) утворюює гострий кут з віссю \(Ox\).
\end{enumerate}
\subsection{Координати вектора в базисі}
\solving


\ansver
\subsection{Вектор, ортогональний до векторів}
\solving


\ansver


\section{Дано вектори\dots}
Дано вектори
\begin{align}
	&\vv{a}=3\vv{p}-2\vv{q},\\
	&\vv{b}=\vv{p}+5\vv{q},
\end{align}
та
\begin{gather}
	\lvert\vv{p}\rvert=4,\\
	\lvert\vv{q}\rvert=\frac{1}{2},\\
	\left(\widehat{\vv{p},\vv{q}}\right)=\frac{5\pi}{6}.
\end{gather}
Знайти:
\begin{itemize}
	\item Площу трикутника побудованого на векторах \(\vv{a}\) і \(\vv{b}\).
	\item Довжини діагоналей паралелограма побудованого на векторах \(\vv{a}\) і \(\vv{b}\).
	\item Проекцію ???
\end{itemize}
\subsection{Площа трикутника}
\solving


\ansver
\subsection{Довжини діагоналей}
\solving


\ansver
\subsection{Проекція}
\solving


\ansver
