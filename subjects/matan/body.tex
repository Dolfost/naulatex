% !TeX root = ../../../document.tex
\nocite{*}
\section{Знайти суму ряду за означенням}
\begin{gather}
	\sum_{n=1}^\infty\frac{7}{49n^2+7n-12} \label{eq:1}
\end{gather}

\solving

Необхідна умова збіжності виконується:
\begin{gather}
	\lim_{n\to\infty} \frac{7}{49n^2+7n-12} \equiv 0
\end{gather}
Розкладемо \(f(x) = \frac{7}{49n^2+7n-12}\) на елементарні дроби:
\begin{gather}
	49n^2+7n-12=0,\ D=49+2352 = 2401,\\
	x_1=\frac{-7+\sqrt{2401}}{98}=\frac{3}{7},\ x_2=\frac{-7-\sqrt{2401}}{98}=-\frac{4}{7},\\
	f(x)=\frac{7}{(x-\frac{3}{7})(x+\frac{4}{7})} = \frac{A}{x-\frac{3}{7}}+\frac{B}{x+\frac{4}{7}},\\
	A\left(x+\frac{4}{7}\right)+B\left(x-\frac{3}{7}\right)=7,\ x=-\frac{4}{7}\Rightarrow B= 7,\ x=\frac{3}{7}\Rightarrow A=7
\end{gather}
Маємо:
\begin{gather}
	f(x)=\frac{7}{x-\frac{3}{7}}+\frac{7}{x+\frac{4}{7}}\Rightarrow \sum_{n=1}^\infty f(x) = \sum_{n=1}^\infty \frac{7}{x-\frac{3}{7}}-\sum_{n=1}^\infty=\\
	=\underbrace{\left(\frac{49}{4}+\underline{\frac{49}{11}+\frac{49}{18}+\dots}\right)}_{\sum_{n=1}^\infty \frac{7}{x-\frac{3}{7}}}-\underbrace{\left(\underline{\frac{49}{11}+\frac{49}{18}+\frac{49}{25}+\dots}\right)}_{\sum_{n=1}^\infty} = \frac{49}{4}.
\end{gather}
\ansver
Сума ряду \ref{eq:1} рівна \(\frac{49}{9}\).

\section{Дослідити на збіжність}
Дослідити на збіжність знакододатний числовий ряд за ознаками порівняння
\begin{gather}
	\sum_{n=3}^\infty \frac{n+5}{(n^2)(n+2)} \label{eq:2}
\end{gather}

\begin{thm}[Гранична ознака порівняння]
	\label{dfn:zbiz}
	Нехай \(\sum_{n=1}^\infty u_n\) і \(\sum_{n=1}^\infty v_n\) -- ряди з додатними членами. Якщо існує скінченна, відмінна від нуля, границя
	\begin{gather}
		\lim_{n\to\infty}\frac{u_n}{v_n}=k\ (0<k<\infty)
	\end{gather}
	то вказані ряди одночасно збіжні або розбіжні.
\end{thm}

\solving
Перетворимо підсумний вираз на еквівалентну функцію:
\begin{gather}
	\frac{n+5}{(n^2)(n+2)}\sim \frac{n}{n^3}\sim \frac{1}{n^2}
\end{gather}
Порівняємо отриману функцію з розбіжним гармонійним рядом \(\sum_{n=1}^\infty\frac{1}{n}\):
\begin{gather}
	\lim_{n\to\infty}\frac{\frac{1}{n}}{\frac{1}{n^2}}=\lim_{n\to\infty}n=\infty \label{eq:waht}
\end{gather}
За теоремою \ref{dfn:zbiz} з границі \ref{eq:waht} випливає що ряд \ref{eq:2} збіжний.
\ansver
Ряд \ref{eq:2} збіжний.
