%!TeX root = ../../../document.tex
\nocite{*}
\sffamily

\section{Постановка задачі}
\addcontentsline{lol}{section}{Постановка задачі}
\bigtext{Тема:} Створення консольного додатку для обчислення всiх простих чисел
до заданого за визначенням.
\bigtext{Мета:}
\begin{itemize}
\item Познайомитися з описом змiнних вбудованих типiв (\code{int}, \code{bool}).
\item Познайомитися з читанням цiлих чисел зi стандартного входу.
\item Познайомитися з виразами, операторами - вираз i операторами
	циклу \code{for} i \code{while}.
\item Познайомитися з можливостями форматування даних i висновком в стандартний вихiд.
\end{itemize}

\section{Теорія}
\begin{dfn}[Просте число]
Просте число  — це натуральне число, яке має рівно два різні натуральні дільники (лише 1 і саме число). Решту чисел, окрім одиниці, називають складеними. Таким чином, всі натуральні числа, більші від одиниці, розбивають на прості і складені. Теорія чисел вивчає властивості простих чисел.
Послідовність простих чисел починається так:
2, 3, 5, 7, 11, 13, 17, 19, 23, 29, 31, 37, 41, 43, 47, 53, 59, 61, 67, 71, 73, 79, 83, 89, 97, 101, 103, 107, 109, 113, 127, 131, 137, 139, 149
\end{dfn}

\section{Про програму}
Програма написана мовою програмування \code{C\#} та реагує лише на перший аргумент комадного рядка - число, яке є верхньою границею пошуку простих чисел серед натуральних починаючи з 1.

\section{Тестування}
\begin{figure}[H]
	\centering
	\includegraphics[width=\textwidth]{images/test}
	\caption{Тестування}
	\label{fig:test}
\end{figure}

\addtocontents{lof}{\bigskip}
\section{Висновок}
Програма працює, при цьому використовуючи час \(O(n\cdot n!)\).