\documentclass[../../../../document]{subfiles}

\begin{document}
	\chapter{Постановка задачі}
	Вміст цієї секції взятий з~\cite{computational_methods}.\\
	\bigtext{Тема:} \worktheme.\\
	\bigtext{Мета:}
	\begin{itemize}
		\item
		\item
	\end{itemize}
	Варіант завдання --- \studentnumber:
	\begin{enumerate}
		\item
			\begin{gather}
			\end{gather}
		\item
			\begin{gather}
			\end{gather}
	\end{enumerate}

	\chapter{Теорія}
	\section{Завдання 1}
	\label{sec:prb1thr}

	\section{Завдання 2}
	\label{sec:prb2thr}

	\chapter{Про програму}
	Програма написана на \textinline|C++| з застосуванням Qt framework.

	Qt (вимовляється як \enquote{к'ют}) - це крос-платформне
	програмне забезпечення для створення графічних інтерфейсів користувача, а також
	крос-платформних додатків, які працюють на різних програмних і апаратних
	платформах, таких як Linux, Windows, macOS, Android або вбудованих системах, з
	невеликими змінами або без змін у базовій кодовій базі, залишаючись при цьому
	нативним додатком з нативними можливостями і швидкістю.
	
	Обчислення математичних виразів у рядковому відображенні відбувається за допомогою бібліотеки \textinline|exptrk| яку написав Arash Partow:
	\begin{itemize}
		\item Офіційна сторінка автора: \url{https://www.partow.net/programming/exprtk/index.html}
		\item Сторінка проекту на \url{github.com}: \url{https://github.com/ArashPartow/exprtk/tree/master}
	\end{itemize}
	\section{Про \texttt{C++}}
	\subsection{Файлова структура програми}
	Програма має наступну структуру:
	\begin{description}
		\item[\textinline{CMakeLists.txt}]
			Набір інструкцій \textinline|CMake| для побудови програми. 
		\item[\textinline{main.cpp}]
			Точка входу в програму. 
		\item[\textinline{calculator.cpp}]
			Імплементація графічного інтерфейсу користувача. 
		\item[\textinline|tabs.cpp|]
			Імплементація обчислень кожної вкладки. Вміст цього файлу зображений на \cref{lst:code}.
		\item[\textinline|exprtk_cmake/|] Бібліотека для обробки математичних виразів, як рядків.
	\end{description}
	До натиску кнопки \enquote{Compute} в кожній вкладці прив'язана відповідна функція з \textinline{tabs.cpp}:
	\begin{longlisting}
		 \begin{Center}
			 % \inputminted{cpp}{\subfix{../../../../../tabs.cpp}}
		 \end{Center}
		 \caption{Файл \textinline{tabs.cpp}}\label{lst:code}
	\end{longlisting}

	\begin{funcDescription}
		\funcitem{cpp}|QString* tab1func(... QString ...)|
			Приймає рядки з введеннями користувача у вкладку 1, повертає буфер символів. 
			Реалізацію цієї функції можна побачити на \cref{lst:code}.
		\funcitem{cpp}|QString* tab2func(... QString ...)|
			Приймає рядки з введеннями користувача у вкладку 2, повертає буфер символів.
			Реалізацію цієї функції можна побачити на \cref{lst:code}.
	\end{funcDescription}
	
	\section{Алгоритм вирішення задачі 1}
	Алгоритм вирішення задачі 1 повністю оснований на теорії з \cref{sec:prb1thr} та реалізує її мовою С++ у функції \cppinline|tab1func()| на \cref{lst:code}.

	\section{Алгоритм вирішення задачі 2}
	Алгоритм вирішення задачі 2 повністю оснований на теорії з \cref{sec:prb2thr} та реалізує її мовою С++ у функції \cppinline|tab2func()| на \cref{lst:code}. Було застосовано \enquote{метод однакових впливів} для обчисленя абсолютних похибок.

	\FloatBarrier
	\chapter{Тестування}
	Протестуємо отриманий прграмний застосунок. 
	\section{Контрольні приклади}
	Перевіримо правильність результатів роботи програми на прикладах які наведені в \cite{computational_methods} у практичному занятті \prodnumber{}.
	\subsection{Приклад 1}
	Приклад 1 має наступну умову:
	\begin{gather}
	\end{gather}
	Та наступну відповідь:
	\begin{gather}
	\end{gather}

	\subsection{Приклад 2}
	Приклад 2 має наступну умову:
	\begin{gather}
	\end{gather}
	Та наступну відповідь:
	\begin{gather}
	\end{gather}
	
	\section{Розвʼязання задачі 1}

	\section{Розвʼязання задачі 2}

	\FloatBarrier
	\chapter{Висновки}
	В результаті розробки програмного застосунку було:
	\begin{itemize}
		\item
		\item
	\end{itemize}
	Було отримано програмний застосунок який здатен:
	\begin{itemize}
		\item
		\item
		\item
		\item
	\end{itemize}
\end{document}
